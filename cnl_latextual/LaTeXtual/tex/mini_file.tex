\documentclass{article}

% 
%tikz graphics
%\usepackage{xcolor} % to remove color.
\usepackage{tikz} % 
\usetikzlibrary{chains,shapes,arrows,%
 trees,matrix,positioning,decorations,fadings}
%\usepackage[framemethod=TikZ]{mdframed}

\def\tikzfig#1#2#3{%
\begin{figure}[htb]%
  \centering
\begin{tikzpicture}#3
\end{tikzpicture}
  \caption{#2}
  \label{fig:#1}%
\end{figure}%
}
\def\smalldot#1{\draw[fill=black] (#1) %
 node [inner sep=1.3pt,shape=circle,fill=black] {}}
\newtheorem{theorem}{Theorem}[subsection]
\newtheorem{lemma}[theorem]{Lemma}
\newtheorem{corollary}[theorem]{Corollary}
\newtheorem{proposition}[theorem]{Proposition}
\newtheorem{definition}[theorem]{Definition}
\newtheorem{example}[theorem]{Example}
\newtheorem{working}[theorem]{Working Hypothesis}

\theoremstyle{remark}
\newtheorem{remark}[equation]{Remark}%[subsection]


\newcommand{\ring}[1]{\mathbb{#1}}
\newcommand{\op}[1]{\hbox{#1}}
\newcommand{\f}[1]{\frac{1}{#1}}
\newcommand{\ang}[1]{\left\langle{#1}\right\rangle}
\def\error{e__r__r__o__r}
\def\cong{\error}
\def\sl{\mathfrak{sl}_2(\ring{R})}
\def\SL{\op{SL}_2(\ring{R})}
\def\SO{\op{SO}_2(\ring{R})}
\def\h{\mathfrak h}
\def\hstar{{\mathfrak h}^\star}
\def\Mstar{M^\star}
\def\D{\ring{D}}
\def\Hcost{H_{cost}}
\def\Hlie{H_{Lie}}
\def\Hh{H_{\h}}
\def\DR{D_{min}}
\def\tmax{t_{max}}

\newcommand\Lsing{\Lambda_{sing}}
\newcommand\lsing{\lambda_{sing}}
\newcommand\ee[1]{e_{#1}^*}
\newcommand{\partials}[2]{\frac{\partial #1}{\partial #2}}

%\newcommand\XX[1]{[XX fix: #1]}
%\newcommand\FIGXX{{\tt [XX Insert figure here]}}
%\newcommand{\rif}[1]{\ref{#1}{\tt-#1}}
%\newcommand{\cat}[1]{\cite{#1}{\tt-#1}}
%\newcommand{\libel}[1]{\label{#1}{\tt(#1)~}}


\title{The Reinhardt Conjecture as an Optimal Control Problem}
\author{Thomas C. Hales\thanks{Research supported by NSF grant
    1104102.  I thank W\"oden Kusner for discussions related to this
    problem.}}  
\date{} 
%\date{February 26, 2017}

\begin{document}

\maketitle


\begin{abstract} 
  In 1934, Reinhardt conjectured that the shape of the centrally
  symmetric convex body in the plane whose densest lattice packing has
  the smallest density is a smoothed octagon.  This conjecture is
  still open.  We formulate the
  Reinhardt Conjecture as a problem in optimal control theory.

  The smoothed octagon is a Pontryagin extremal trajectory with
  bang-bang control.  More generally, the smoothed regular $6k+2$-gon
  is a Pontryagin extremal with bang-bang control.  The smoothed
  octagon is a strict (micro) local minimum to the optimal control
  problem.
  
  The optimal solution to the Reinhardt problem is a trajectory
  without singular arcs.  The extremal trajectories that do not meet
  the singular locus have bang-bang controls with finitely many
  switching times.

  Finally, we reduce the Reinhardt problem to an optimization problem
  on a five-dimensional manifold.  (Each point on the manifold is an
  initial condition for a potential Pontryagin extremal lifted
  trajectory.)  We suggest that the Reinhardt conjecture might
  eventually be fully resolved through optimal control theory.

  Some proofs are computer-assisted using a computer algebra system.
\end{abstract}

\parskip=0.8\baselineskip
\baselineskip=1.05\baselineskip

\newenvironment{blockquote}{%
  \par%
  \medskip%
  \baselineskip=0.7\baselineskip%
  \leftskip=2em\rightskip=2em%
  \noindent\ignorespaces}{%
  \par\medskip}

\section{Introduction}

In 1934, Reinhardt conjectured that the shape of centrally symmetric
body in the plane whose densest lattice packing has the smallest
density is a smoothed octagon (Figure \ref{fig:octagon}).  The
corners of the octagon are rounded by hyperbolic arcs.  For popular
accounts of the Reinhardt conjecture, including some spectacular
animated graphics by Greg Egan, see \cite{baez-egan}, \cite{baez}.

\tikzfig{octagon}{ A smoothed octagon is conjectured to have the worst
  best packing among centrally symmetric disks in the plane.  }{
% data generate in Mathematica plotOct:
\draw (1, 0) --  (1., 0.229378) --  (0.969314, 0.403215) --  (0.839864, 
  0.544075) --  (0.659754, 0.714692) --  (0.5, 0.866025);
\draw (0.5, 0.866025) --  (0.340246, 0.944071) --  (0.129449, 
  0.94729) --  (-0.129449, 0.94729) --  (-0.340246, 0.944071) --  (-0.5, 
  0.866025);
\draw (-0.5, 0.866025) --  (-0.659754, 0.714692) --  (-0.839864, 
  0.544075) --  (-0.969314, 0.403215) --  (-1., 0.229378) --  (-1., 0);
\draw (-1., 0) --  (-1., -0.229378) --  (-0.969314, -0.403215) --  (-0.839864,  
-0.544075) --  (-0.659754, -0.714692) --  (-0.5, -0.866025);
\draw (-0.5, -0.866025) --  (-0.340246, -0.944071) --  (-0.129449, -0.94729) --   
(0.129449, -0.94729) --  (0.340246, -0.944071) --  (0.5, -0.866025);
\draw (0.5, -0.866025) --  (0.659754, -0.714692) --  (0.839864, -0.544075) --   
(0.969314, -0.403215) --  (1., -0.229378) --  (1., 0);
}


This article is a continuation of an article from 2011, which formulates
the Reinhardt conjecture as a problem in the calculus of variations
\cite{hales2011reinhardt}.  This article reformulates the Reinhardt
conjecture as an optimal control problem.
% (\url{https://arxiv.org/abs/1103.4518})

Bang-bang controls of an optimal control problem are controls that
switch between extreme points of a convex control set (often with a
finite number of switches).  A major theme of optimal control is the
study of bang-bang controls, and the extremal trajectories of many
control problems have bang-bang controls.  Intuitively, bang-bang
controls switch from one extreme position to another: navigating a
craft by flooring the accelerator pedal then slamming on the brakes;
or steering a vehicle by making the sharpest possible turns to the
left and to the right; or maximizing wealth by investing all resources
in a single financial asset for a time, then suddenly moving all
resources elsewhere.

\end{document}
